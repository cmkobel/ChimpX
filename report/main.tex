\documentclass[fleqn,12pt]{wlscirep}
\usepackage[utf8]{inputenc}
\usepackage[T1]{fontenc}

% Carls packages
\usepackage{wrapfig}

\title{Copy Number Variation in The Chimpanzee X Chromosome}

\author{Project in Bioinformatics, 5 ECTS, Fall 2019 - Carl Mathias Kobel}



\begin{abstract}
Counting the copy number of genes in an individual can be done by read mapping to a reference, and measuring the coverage of each gene. Normalizing this measure, such that it can be compared across individuals can be done either by scaling with the coverage of a long single-copy gene (i.e. DMD) or by assuming that the copy number distribution for all linked genes is equal to one. In this study, both methods are discussed, and the latter method is used for a search of genes, on the Chimpanzee X chromosome, with high copy number variation.
It is observed that a number of previously undescribed genes with high copy number and which show signs of testis-expression, might be part of a meiotic drive process.
Anyhow, because expression data on Chimpanzee genes is in many occasions not readily available, it is not possible to immediately conclude that any of these genes are indeed related to meiotic drive.
\end{abstract}
%\setlength{\parindent}{0pt}

\begin{document}

\flushbottom
\maketitle


\thispagestyle{empty}


\section*{Introduction}






The primate X chromosome has undergone extensive adaptation to become part of the sexual differentiation mechanism now present in primates \cite{skaletsklyMSY}.  It is proposed that the X and Y chromosomes are divergent copies of a once present autosome. The Y chromosome has lost most of its genes and now most importantly hosts the testis determining factor (SRY), which defines the sex of the primate individual. The other genes still present on the Y chromosome have mostly become inactivated and are now referred to as the X-degenerate region \cite{skaletsklyMSY}.


It is suggested that the X and Y chromosomes might interfere with one another during male meiosis where either the X or Y chromosome is passed down to the sperm cell, such that the corresponding sex chromosome has a higher probability of being transmitted to the offspring. This process is termed meiotic drive generally refers to an unequal segregation of sex chromosomes from the heterogametic sex. This results in biased sex-ratios in the population, and also implies that the fitness of these chromosomes is not optimal.?? formuler bedre \cite{doi:jaenike_10.1146/annurev.ecolsys.32.081501.113958}. It is hypothesized that this tug of war between the two sex-chromosomes, is played out by incrementing in the copy number of drive-related genes on each chromosome. This means that X chromosomes containing drive coding genes might have developed corresponding drive-suppressing genes on the Y chromosome. 

The search for genes, or underlying regions, which are part of this meiotic drive process, can be obtained by looking for genes with a high copy number variation between individuals. If we can assess the copy number variation of the genes on the X chromosome, we can investigate these genes expression in testis, and possible further hypothesize that these genes are part of a meiotic drive process.

??Meiotic drive processes are studied in relation to hybrid incompability??ref


In a more or less recent study\cite{Lucotte907}, ampliconic regions in human have been investigated in their relation to testis expression, and here we present a smaller scaled but otherwise similar study, where we look for ampliconic regions on the Chimpanzee X chromosome.



?? kig på artikler der linker til lucotte 2018





\section*{Methods and Project Process}
In order to measure the CNV of the genes on the Chimpanzee X chromosome, we aligned the reference chromosome X (Pan\_Tro\_3) to itself and created dotplots in partially overlapping windows of 500 Kbp. These dotplots depict internal duplicates inside and immediately surrounding the sequence in these windows. By manually browsing the catalog of 500 overlapping windows, we decided manually for each of the $\sim$1000 genes in the chromosome if they showed enough internal duplicates to be included in the downstream analysis. By concatenating these selected genes into an artificial chromosome (AC), and then mapping the reads from each individual to this AC, we were able to measure the relative CNV of the genes. The absolute coverage here was normalized using DMD, which is a long single-copy gene. Because the dotplot method takes only adjacent duplicates into account (limited window size), this method makes it possible to identify only the duplicated regions that reside inside and near by genes in the window. As a solution to this problem, we decided instead to  compute the CNV of all annotated genes in the chromosome. This was achieved by mapping the reads from each individual to a well annotated reference chromosome (Pan\_tro\_3). To get a relative measure of CNV for each gene we again normalized using the DMD gene. Mapping to all X-linked genes might be more computationally intensive than mapping to only a few, but it has the advantage of eliminating selection bias. Another advantage is, that reads which map better in paralogous pseudogenes, outside the selected genes, will not affect the coverage of the annotated genes. We filtered to retain only the reads that satisfy the following conditions: only primary mapped reads, a mapping quality of at least 50, a consecutive mapping of at least 100 bp and an overall maximum nucleotide mismatch of at most 2. Because some individuals, namely Simliki and Julie-A959, showed large deletions in the DMD gene, we decided to assume that the median of the copy numbers of all genes in the chromosome should equal 1. Accepting this assumption, it was now possible to keep Simliki and Julie-A959 in the analysis. Instead of normalizing the coverage with the coverage of DMD, we normalized by shifting the median chromosome wide coverage for each individual to one.

The read mapping method is inspired from Lucotte \textit{et al.} 2019\cite{Lucotte907}, where a similar study was performed.

This study uses publicly available Chimpanzee genomes. See table \ref{tab:subjects} All data was sourced from the EMBL-EBI European Nucleotide Archive.



\section*{Results and discussion} % Kan også kaldes Results and Discussion. Og så kan Discussion (sek. 4 kaldes conclusion)


\subsection*{Overall distribution}
The overall distribution of copy numbers, across all genes and individuals, is shown in figure \ref{fig:qc_dist_all}. The distribution is near symmetrical, centered around a mean of 1.

% Figure 1
\begin{figure}[h] 
  \centering
  \includegraphics[width=0.58\textwidth]{figures/fig_qc_dist_all_1.pdf}
  
  \caption{Distribution of median CN across all genes for all subjects not including Simliki. A normal distribution with an equal mean, SD and area (grey) is overlapped as a visual aid. Colored by sex.}
  
  \label{fig:qc_dist_all}
\end{figure}

\noindent There is no known prior on the chromosome wide CN distribution. But a mean close to one makes good sense, as it suggests that linked paralogs have been described separately in the annotation. It also suggests, that the genes in the annotation have a representative length.

We checked that all individuals obey this distribution (figure \ref{fig:qc_subjects_simliki}), and observed that most individuals do. But, Simliki shows a chromosome wide CN median of 1.64, which suggests that Simliki's DMD gene has a size of $ \frac{100\%}{1.64} = 61\%} $ compared to the reference annotation. Failure to exclude Simliki from subsequent analysis might inflate the CN measurement. Thus, we decided to remove Simliki from subsequent analysis, and hypothesize, that Simliki's DMD gene might have a deletion. All other individuals have a chromosome wide median CN of less than 1.5, and are thus kept in the downstream analysis.??(fjernelse nævnt to gange) ??overvej også at teste forskel på køn med parret permutationstest.

Comparing the genes with the highest copy number variation with the results of Lucotte et al





% Figure 2
\begin{figure}[h] 
  \centering
  \includegraphics[scale=0.78]{figures/fig_qc_subjects_simliki_1.pdf}
  \caption{Distribution of median CN across all genes for each subject. Box edges denote quartiles. Points denote values beyond 1.5*IQR, where IQR is the interquartile range.}
  \label{fig:qc_subjects_simliki}
\end{figure}



%Some of the genes with the highest median copy numbers are overlapping. This spurred the hypothesis that 
\noindent Of the 919 genes on the X chromosome, 64 ({\raise.17ex\hbox{$\scriptstyle\sim$}}7\%) have an overlap with another gene. Performing a non-parametric (permutation) test on the difference (in means) of the distributions of medians for overlapping and non-overlapping genes, shows that the difference is not significantly different (\textit{p}-value = 0.1927).
Two genes, {\footnotesize CT55} and {\footnotesize TCP11x2} are duplicated in full length on the annotation, but since they each have relatively low variances (2nd quartile of overall SD), they are not considered actively ampliconic.

\subsection*{Checking paralogs on autosomes}
In this study, we map the reads to the X chromosome only. Because the rest of the chromosomes might contain paralogs of the genes on the X chromosome, it is relevant to check if the exclusion of the rest of the genome has a significant impact on the coverage of the genes on the X chromosome. In order to check this, we randomly picked a number of individuals (?? which) and compared the coverage of the X chromosome genes from this read mapping, to the read mapping of only the X chromosome. ??results




\subsection*{Most copy number-variant genes}


% Figure 3
\begin{figure}[h] 
  \centering
  \includegraphics[scale=0.78]{figures/fig_main_median_3.pdf}
  \caption{Copy number from all subjects, grouped by genes. These 27 genes or overlaps of genes have a median copy number $\geq$ 1.5. Sorted by descending median. '\&' signifies the overlap of two genes. Horizontal jitter is applied. }
  \label{fig:fig_main}
\end{figure}


We looked up the genes with the highest median copy numbers. Unfortunately, many of the genes in the Pan\_tro\_3 reference genome are not named as exhaustively as in the human counterparts, and many of the genes thus have serial numbers instead. These serial numbers will be abbreviated as ENSPTRG00000049971 to E..49971. 
The genes with the highest median copy numbers are enumerated in figure \ref{fig:fig_main}.
\noindent E..49971, E..42923, E..46688,  E..22234, E..50351, E..49979, E..46615, E..49911, E..47182, E..48376, E..50450, E..49942, POLA1 have no obvious relation to anything meiosis related by name or description, and there is no expression data available. E..22357, XAGE5, E..22336,  is testis expressed but has no other description. E..21637 and OPN1LW have expression in several organs, where testis has the highest. EDA, FGF13 and E..41347 is expressed in several organs as well as the testis. TCP11X2, E..23212, TEX28 are testis expressed but has no other description. For E..48802, no Chimpanzee data is related, but an mRNA of an ortholog in Macaca Fascicularis is testis expressed. SPANXN5 has no expression data, but its description "Sperm protein associated with the nucleus on the X chromosome N1" suggests that it is expressed in the testis. ETDB has an identical homolog which is termed "Embryonic testis differentiation homolog", and thus suggests testis expression. E..52876 and E..16737 have no Chimpanzee expression data, but an ortholog in human with an identity of 90\% is testis expressed. E..28324 is not present in all subjects for unknown reasons. Nevertheless, has a median above 1.5, and is testis expressed.
\noindent Of the genes described in Lucotte et al. 2019, only {\footnotesize OPN1LW} is overlapping as a copy number-variant gene. Even though it codes for long-wavelength opsin in the retina, it has its highest expression in the testis according to the Bgee database (in Chimpanzee). Of the other genes in deemed copy number-variant Lucotte et al. 2019 - CT47A, presumably orthologous to ENSPTRG00000046894 "cancer/testis antigen 47A" has a max CN of 1.8 and a median CN of 0.9, and is thus not as variant in chimpanzee as in human (max CN: 15.07, median CN: 5.01). The other genes deemed copy number-variant in Lucotte et al. 2019 do not have identically named genes in the Chimpanzee annotation.
That is, assuming that the naming of the genes across the species is completely parsimonous. An alignment is necessary to be able to conclude that these are in fact orthologs.
\\



\noindent It seems like there is a relation between the high median CN and testis presence. As we don't know what proportion of the X-linked genes are expressed in the testis, it is hard to say if the genes with the highest variation are significantly more present in the testis. Nevertheless, for the genes where expression data was available, all genes but IL1RAPL2 showed signs of presence in the testis.






\\
\\
\noindent ??flyt højere op: Assuming that the genome wide CN median should be 1, suggests that it should be possible to convert the relative copy number measures to absolute by normalizing the subject-wise median to 1. Nonetheless, the DMD gene is a promising normalization candidate, as most subject get a median close to 1.

\\
\\
\subsection*{Copy number variation among relatives}
In order to investigate the copy number turnover between parents-offspring, we plotted the copy number over time measured in n'th degree relatives. For 7 of the 33?? Chimpanzees included in this study, we know the parent relation, and for 3 we know the grandparent relation. This means that we can compare the progression of copy numbers, down through the pedigree. In figure \ref{fig:pedigree_CN}, the copy numbers for the 5 most highly copied genes as well as OPN1LW is visualized. Dad-son relations are removed because no X chromosome is passed in this relation.

% Figure 4
\begin{figure}[h] 
  \centering
  \includegraphics[scale=0.78]{figures/fig_pedigree_CN_3.pdf}
  \caption{Copy number as a function of pedigree. Each straight line denotes the change in copy number for an X-linked gene, when the X chromosome is passed from parent to offspring. Dad-son relations are not included.}
  \label{fig:pedigree_CN}
\end{figure}






\section*{Conclusion}

This study has searched for genes related to meiotic drive and found handful of good candidates. For the genes where expression data is present, it seems like there is a relation between high median copy number and testis presence. This might support the hypothesis that there is a relation between CNV and meiotic drive.

There is a clustering of median between sexes in some of the most ampliconic genes. The tendency is that males have a higher copy number than females. Though this might be a signal, nothing has been concluded or hypothesized from it. Whether it stems from reads from the X degenerate region on the Y chromosome can be investigated be concatenating the X and Y chromosomes together and checking whether the clustering vanishes. As a test, one of the most sexually grouped genes (E..46688) was mapped to the Y chromosome with no matches, which suggests that the sex-grouping is not due to methodological errors. A different and much more interesting act, would be to propose that there is a link between high copy numbers of these genes, and the segregation of a Y chromosome to the sperm cell, leading to male individuals with high copy numbers. It might also be due to random fluctuations.

E..22234, E..21637 and ?? might be good candidates of genes involved in meiotic drive processes. Expression data is missing on the three most wildly copied genes (E..49971, E.46688, E..42923), which might also have a testis-relation.

For many of the genes where identical sequnces are found in other species (mostly human) where testis expression is present, it is not yet investigated whether the ampliconic region is identical to the ortholog or not.

It would be appropiate to identify the ampliconic regions on each of the copy number variant genes. This can be done with dotplots. In order to investigate the cause of the amplification, it would make sense to identify SNPs on these regions, and investigate whether these SNPs are related to amplification??.

This study has found a handful of genes with high copy number variation which are expressed in the testis. Though some of the genes had no description or expression data available, it seems like most of the most copy variant genes are testis related.
\\
\\
\\

?? Relate to meiotic drive. Give ideas to further analysis.

?? Læs bachproj igen og få ideer.

?? fejlkilder

??Using decoy genome to get rid of paralogs.




\bibliography{sample}

%\newpage
\section*{Appendix}

\begin{table}[h]
\centering
\begin{tabular}{|l|l|l|l|}
\hline
\textbf{Subject} & \textbf{Sex} & \textbf{Species} & \textbf{Source} \\
\hline
 

Julie-LWC21  &  female  &  Pan troglodytes ellioti  & Prado-Martinez \textit{et al.} 2013 \cite{pradomartinezgagp}  \\
Akwaya-Jean  &  male  &  Pan troglodytes ellioti  & Prado-Martinez \textit{et al.} 2013 \cite{pradomartinezgagp}  \\
%Banyo  &  female  &  Pan troglodytes ellioti  & %Prado-Martinez \textit{et al.} 2013 \cite{pradomartinezgagp}  \\
%Basho  &  male  &  Pan troglodytes ellioti  & %Prado-Martinez \textit{et al.} 2013 \cite{pradomartinezgagp}  \\
Damian  &  male  &  Pan troglodytes ellioti  & Prado-Martinez \textit{et al.} 2013 \cite{pradomartinezgagp}  \\
%Kopongo  &  female  &  Pan troglodytes ellioti  & %Prado-Martinez \textit{et al.} 2013 \cite{pradomartinezgagp}  \\
Koto  &  male  &  Pan troglodytes ellioti  & Prado-Martinez \textit{et al.} 2013 \cite{pradomartinezgagp}  \\
%Paquita  &  female  &  Pan troglodytes ellioti  & %Prado-Martinez \textit{et al.} 2013 \cite{pradomartinezgagp}  \\
Taweh  &  male  &  Pan troglodytes ellioti  & Prado-Martinez \textit{et al.} 2013 \cite{pradomartinezgagp}  \\
%Tobi  &  female  &  Pan troglodytes ellioti  & %Prado-Martinez \textit{et al.} 2013 \cite{pradomartinezgagp}  \\
Andromeda  &  female  &  Pan troglodytes schweinfurthii  & Prado-Martinez \textit{et al.} 2013 \cite{pradomartinezgagp}  \\
Harriet  &  female  &  Pan troglodytes schweinfurthii  & Prado-Martinez \textit{et al.} 2013 \cite{pradomartinezgagp}  \\
Kidongo  &  female  &  Pan troglodytes schweinfurthii  & Prado-Martinez \textit{et al.} 2013 \cite{pradomartinezgagp}  \\
Nakuu  &  female  &  Pan troglodytes schweinfurthii  & Prado-Martinez \textit{et al.} 2013 \cite{pradomartinezgagp}  \\
Bwambale  &  male  &  Pan troglodytes schweinfurthii  & Prado-Martinez \textit{et al.} 2013 \cite{pradomartinezgagp}  \\
Vincent  &  male  &  Pan troglodytes schweinfurthii  & Prado-Martinez \textit{et al.} 2013 \cite{pradomartinezgagp}  \\
Clara  &  female  &  Pan troglodytes troglodytes  & Prado-Martinez \textit{et al.} 2013 \cite{pradomartinezgagp}  \\
Doris  &  female  &  Pan troglodytes troglodytes  & Prado-Martinez \textit{et al.} 2013 \cite{pradomartinezgagp}  \\
Julie-A959  &  female  &  Pan troglodytes troglodytes  & Prado-Martinez \textit{et al.} 2013 \cite{pradomartinezgagp}  \\
Vaillant  &  male  &  Pan troglodytes troglodytes  & Prado-Martinez \textit{et al.} 2013 \cite{pradomartinezgagp}  \\
Jimmie  &  female  &  Pan troglodytes verus  & Prado-Martinez \textit{et al.} 2013 \cite{pradomartinezgagp}  \\
Bosco  &  male  &  Pan troglodytes verus  & Prado-Martinez \textit{et al.} 2013 \cite{pradomartinezgagp}  \\
Clint  &  male  &  Pan troglodytes verus  & Prado-Martinez \textit{et al.} 2013 \cite{pradomartinezgagp}  \\
Koby  &  male  &  Pan troglodytes verus  & Prado-Martinez \textit{et al.} 2013 \cite{pradomartinezgagp}  \\
Donald  &  male  &  Pan troglodytes verus x troglodytes & Prado-Martinez \textit{et al.} 2013 \cite{pradomartinezgagp}  \\
Carolina  &  female  &  Pan troglodytes verus  & Besenbacher \textit{et al.} 2018 \cite{Besenbacher2018DirectEO}\\
Simliki  &  female  &  Pan troglodytes verus  & Besenbacher \textit{et al.} 2018 \cite{Besenbacher2018DirectEO}  \\
Carl  &  male  &  Pan troglodytes verus  & Besenbacher \textit{et al.} 2018 \cite{Besenbacher2018DirectEO}  \\
Frits  &  male  &  Pan troglodytes verus  & Besenbacher \textit{et al.} 2018 \cite{Besenbacher2018DirectEO}  \\
Pearl  &  female  &  Pan troglodytes verus  & Venn \textit{et al.} 2014\cite{Venn1272} \\
Marlies  &  female  &  Pan troglodytes verus  & Venn \textit{et al.} 2014\cite{Venn1272}  \\
Marco  &  male  &  Pan troglodytes verus  & Venn \textit{et al.} 2014\cite{Venn1272}  \\
Dirk1  &  male  &  Pan troglodytes verus  & Venn \textit{et al.} 2014\cite{Venn1272}  \\
Dennis  &  male  &  Pan troglodytes verus  & Venn \textit{et al.} 2014\cite{Venn1272}  \\
Ruud  &  male  &  Pan troglodytes verus  & Venn \textit{et al.} 2014\cite{Venn1272}  \\
Dylan  &  male  &  Pan troglodytes verus  & Venn \textit{et al.} 2014\cite{Venn1272}  \\
Marlon  &  male  &  Pan troglodytes verus  & Venn \textit{et al.} 2014\cite{Venn1272}  \\
Pat  &  male  &  Pan troglodytes verus  & Venn \textit{et al.} 2014\cite{Venn1272} \\
\hline

\end{tabular}
\caption{\label{tab:subjects}Metadata for the 33 subjects used in this study.}
\end{table}











\end{document}