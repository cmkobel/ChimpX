\section*{Conclusion}

This study has searched for genes related to meiotic drive and found handful of good candidates. For the genes where expression data is present, it seems like there is a relation between high median copy number and testis presence. This might support the hypothesis that there is a relation between CNV and meiotic drive.

There is a clustering of median between sexes in some of the most ampliconic genes. The tendency is that males have a higher copy number than females. Though this might be a signal, nothing has been concluded or hypothesized from it. Whether it stems from reads from the X degenerate region on the Y chromosome can be investigated be concatenating the X and Y chromosomes together and checking whether the clustering vanishes. A different and much more interesting act, would be to propose that there is a link between high copy numbers of these genes, and the segregation of a Y chromosome to the sperm cell, leading to male individuals with high copy numbers. It might also be due to random fluctuations.

E..22234, E..21637 and ?? might be good candidates of genes involved in meiotic drive processes. Expression data is missing on the three most wildly copied genes (E..49971, E.46688, E..42923), which might also have a testis-relation.

For many of the genes where identical sequnces are found in other species (mostly human) where testis expression is present, it is not yet investigated whether the ampliconic region is identical to the ortholog or not.

This study has found a handful of genes with high copy number variation which are expressed in the testis. Though some of the genes had no description or expression data available, it seems like most of the most copy variant genes are testis related.
\\
\\
\\

?? Relate to meiotic drive. Give ideas to further analysis.

?? Læs bachproj igen og få ideer.

?? fejlkilder

??Using decoy genome to get rid of paralogs.


