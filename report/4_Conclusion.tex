\section*{Conclusion}

This study has searched for genes related to meiotic drive and found handful of good candidates. For the genes where expression data is present, it seems like there is a relation between high median copy number and testis presence. Assuming that there is a relation between copy number variation and meiotic drive, these genes are candidates for being involved in this process.

There is a clustering of median between sexes in some of the most ampliconic genes. The tendency is that males have a higher copy number than females. Though this might be a signal, nothing has been concluded or hypothesized from it. Whether it stems from reads from the X degenerate region on the Y chromosome can be investigated be concatenating the X and Y chromosomes together and contrasting the copy number variation between the sexes. As a test, one of the most sexually grouped genes (E..46688) was mapped to the Y chromosome with no matches, which suggested that the sex-grouping is not due to methodological errors. A different and much more interesting act, would be to propose that there is a link between high copy numbers of these genes, and the segregation of a Y chromosome to the sperm cell, leading to male individuals with high copy numbers. It might also be due to random fluctuations, which is supported by the unequal sex ratios of each subspecies.


E..50351, E..22234, E..21637, E..22336, E48802 and XAGE5 might be good candidates of genes involved in meiotic drive processes. Unfortunately, expression data is missing on the three most widely copied genes (E..49971, E..42923, E.46688).

For many of the genes where identical sequences are found in other species (mostly human) where testis expression is present, it is not yet investigated whether the ampliconic region of these genes highly identical to the ortholog or not.

It would be appropriate to identify the ampliconic regions on each of the copy number variant genes. This can be done with dotplots. In order to investigate the cause of the amplification, it would make sense to identify SNPs on these regions, and investigate whether these SNPs are related to amplification.

The most ampliconic genes found in this study does not overlap with the genes with the most copy number-variant genes in Lucotte \textit{et al.} 2018. This suggests that the ampliconic behaviour in all these genes has occurred after the Human-Chimpanzee speciation event.

This study has found a handful of genes with high copy number variation which are expressed in the testis. Though some of the genes had no description or expression data available, it seems like most of the most copy variant genes are testis related.

