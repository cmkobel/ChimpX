\section*{Introduction}


The primate X chromosome has undergone extensive adaptation to become part of the sexual differentiation mechanism now present in primates \cite{skaletsklyMSY}.  It is proposed that the X and Y chromosomes are divergent copies of a once present autosome. The Y chromosome has lost most of its genes and now most importantly hosts the testis determining factor (SRY), which defines the sex of the primate individual. The other genes still present on the Y chromosome have mostly become inactivated and are now referred to as the X-degenerate region \cite{skaletsklyMSY}.


It is suggested that the X and Y chromosomes might interfere with one another during male meiosis where either the X or Y chromosome is passed down to the sperm cell, such that the corresponding sex chromosome has a higher probability of being transmitted to the offspring. This process termed meiotic drive generally refers to an unequal segregation of sex chromosomes from the heterogametic sex. This results in biased sex-ratios in the population, and also implies that the fitness of these chromosomes is not optimal \cite{doi:jaenike_10.1146/annurev.ecolsys.32.081501.113958}. It is hypothesized that this arms race between the two sex-chromosomes, is played out by incrementing in the copy number of drive-related genes on each chromosome. This means that X chromosomes containing drive coding genes or regions might have developed corresponding drive-suppressing genes on the Y chromosome. 

There is evidence for such an intragenomic conflict in mice, in a pair of homologous genes (Sly and Slx) residing on each of the sex chromosomes \cite{SOH2014800}. This pair of genes compete to be transmitted to the next generation. A deficiency of Slx distorts the sex ratio to have higher frequency of males, and vice-versa for Sly. An intragenomic conflict might be the cause behind a speciation event, because the variation that is generated through the arms race might render sex chromosomes, that have been isolated, incompatible.

The search for genes, or underlying regions, which are part of this meiotic drive process, can be obtained by looking for genes with a high copy number variation between individuals. If we can assess the copy number variation of the genes on the X chromosome, we can investigate these genes' expression in testis, and possible further hypothesize that these genes are part of a meiotic drive process.

In a more or less recent study\cite{Lucotte907}, ampliconic regions in human have been investigated in their relation to testis expression, and here we present a smaller scaled but otherwise similar study, where we look for ampliconic regions on the Chimpanzee X chromosome.



%?? Se hvad der linker til lucotte.