\section*{Introduction}



The primate X chromosome has undergone extensive adaptation to become part of the sexual differentiation mechanism. Its presence, and thereby possible displacement of the testis determining factor (SRY) defines the sex of the primate individual. This means that the genes on the X chromosome might interfere with the genes on the Y chromosome in a phenomenon termed meiotic drive. ??ref suggests that there is a link between copy number variation and meiotic drive. It is hypothesized that this tug of war, between the two sex-chromosomes, is played out by incrementing in the copy number of drive-related genes on each chromosome. This means, that if we can assess the copy number (CN) variation of the genes, we can hypothesize that the genes with a highly varying copy number, to be part of a meiotic drive process. Meiotic drive is studied in its relation to hybrid incompatibility ??ref.
