\section*{Methods and Project Process}
In order to measure the CNV of the genes on the Chimpanzee X chromosome, we aligned the reference chromosome X (Pan\_Tro\_3) to itself and created dotplots in partially overlapping windows of 500 Kbp. These dotplots depict internal duplicates inside and immediately surrounding the sequence in these windows. By manually browsing the catalog of 500 overlapping windows, we decided manually for each of the ~1000 genes in the chromosome if they showed enough internal duplicates to be included in the downstream analysis. By concatenating these selected genes into an artificial chromosome (AC), and then mapping the reads from each individual to this AC, we were able to measure the relative CNV of the genes. The absolute coverage here was normalized using DMD, which is a long single-copy gene. Because the dotplot method takes only adjacent duplicates into account (limited window size), this method makes it possible to identify only the duplicated regions that reside inside and near by genes in the window. As a solution to this problem, we decided instead to  compute the CNV of all annotated genes in the chromosome. We did this by mapping the reads from each individual to a well annotated reference chromosome (Pan\_tro\_3). To get a relative measure of CNV for each gene we again normalized using the DMD gene. Mapping to all X-linked genes might be more computationally intensive than mapping to only a few, but it has the advantage of eliminating selection bias. Another advantage is, that reads which map better in paralogous pseudogenes, outside the selected genes, will not affect the coverage of the annotated genes. We filtered to retain only the reads that satisfy the following conditians: only primary mapped reads, a mapping quality of at least 50, a consecutive mapping of at least 100 bp and an overall maximum nucleotide mismatch of at most 2. Because this project uses publicly available data from different projects, the quality and overall coverage might differ (see table \ref{tab:subjects}). Because some individuals, namely Simliki and Julie-A959, showed large deletions in the DMD gene, we decided to assume that the median of the copy numbers of all genes in the chromosome should equal 1. Based on this assumption, it was now possible to keep Simliki and Julie-A959 in the analysis.

The method is inspired from Lucotte et al. 2019\cite{Lucotte907}, where a similar study performed.